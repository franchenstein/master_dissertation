\vspace{0.2cm} Agrade�o em primeiro lugar a Deus por me dar capacidade e ferramentas para realizar este trabalho, pois sem Ele sei que n�o seria poss�vel conclu�-lo. � minha fam�lia pelo apoio, 
compreens�o, carinho e apoio incondicional nas diversas etapas e dificuldades deste trabalho. Em especial � minha m�e e meu pai, por sempre se preocupar em oferecer o melhor poss�vel para seus 
filhos. Ao meus orientadores, professor Cecilio Jos� Lins Pimentel e o professor Daniel Pedro Bezerra Chaves, por sua orienta��o, paci�ncia, dedica��o e esfor�o em sempre me guiar pelo melhor 
caminho, porque sem eles a realiza��o deste trabalho n�o seria poss�vel. Aos meus amigos, pelo apoio nos momentos de dificuldades. Em especial gostaria de agradecer a Paulo Hugo e 
Carlos Eduardo, meus grandes amigos, pelo apoio espiritual e por sempre que poss�vel se dispor a me ajudar nos diversos assuntos te�ricos, pelas observa��es ao longo do desenvolvimento do 
trabalho, al�m de sempre me dar for�a. Aos todos os estudantes do LACRI pelos conselhos e dicas na etapa final deste trabalho. Aos professores do DES com os quais formei a base do meu 
aprendizado por meio de diversas cadeiras. Aos funcion�rios do DES, em especial � Andr�a Ten�rio e Rog�rio F�lix, trabalhadores da secret�ria do Programa de P�s Gradua��o em Engenharia 
El�trica (PPGEE) por sua compet�ncia, organiza��o e sempre disponibilidade em me ajudar nos diversos momentos. Por fim � Coordena��o do PPGEE e ao Conselho Nacional de Desenvolvimento 
Cient�fico e Tecnol�gico (CNPq) pelo suporte financeiro ao longo do desenvolvimento desta disserta��o. 
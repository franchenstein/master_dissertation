\chapter{Revision}\label{cap:2}


{\lettrine[loversize=0.25,findent=0.2em,nindent=0em]{T}{his} chapter revises the concepts needed to develop the algorithms and their application.

\section{Sequences of Discrete Symbols}
This section provides tools to describe sequences of discrete symbols. The length of a sequence \textit{u} is denoted by $|\textit{u}|$. The empty sequence $\epsilon$ is defined as the sequence with length 0. The set of all possible symbols for a sequence is called its alphabet, represented as $\Sigma$. The set of all possible sequences of \textit{n}, \textit{n} $in \mathbb{Z}$ symbols from $\Sigma$ is $\Sigma^n$ and the set of all sequences of symbols from $\Sigma$ with all possible lengths, including the empty sequence sequence $\epsilon$, is $\Sigma^*$. 

Two sequences \textit{u} and \textit{v} $\in \Sigma^*$ can be concatenated to form a sequence \textit{uv}. For example, using a binary alphabet $\Sigma = \{0,1\}$ and \textit{u} = 1010 and \textit{v} = 111, they can be concatenated to form \textit{uv} = 1010111. 

Note that $|\textit{uv}| = |\textit{u}| + |\textit{v}|$. Concatenation is associative, which means \textit{u}(\textit{vw}) = (\textit{uv})\textit{w} = \textit{uvw}, but it is not commutative, as \textit{uv} is not necessarily equal to \textit{vu}. This means that $\Sigma^*$ with the operation of concatenation is a Monoid, as it is a set with an associative operation with an identity element (the empty string).

A sequence \textit{v} $\in \Sigma^*$ is called a suffix of a sequence \textit{w} $\in \Sigma^*$ (given $|\textit{w}| > |\textit{v}|$) if \textit{w} can be written as a concatenation \textit{uv}, where \textit{u} $\in \Sigma^*$, that is \textit{w} = \textit{uv}. In this same sense, the sequence \textit{u} is called a prefix of \textit{w}. 

\section{Graphs}

\begin{definition}[\textbf{Graph}]\label{def:graph}
A graph \textit{G} over the alphabet $\Sigma$ consists of a triple (\textit{Q}, $\Sigma$, $\delta$):
\begin{itemize}
	\item \textit{Q} is a finite set of states with cardinality $|Q|$;
    \item $\Sigma$ is a finite alphabet with cardinality $|\Sigma|$;
    \item $\delta$ is the state transition function $Q\times\Sigma \rightarrow Q$;
\end{itemize}
\end{definition}

Each state \textit{q} from \textit{Q} has at most $|\Sigma|$ outgoing edges. Each outgoing edge from a state is labeled with a unique symbol from $\Sigma$ and it arrives at only one state \textit{$q^{\prime}$} $\in$ \textit{Q}. This behavior is described by the function $\delta : Q \times \Sigma \rightarrow Q$, which is the transition function. For example, leaving state \textit{q} with the edge labeled with \textit{a} and arriving at state \textit{$q^{\prime}$} is represented by $\delta(q, 0) = q^{\prime}$. Figure \ref{fig:graph} shows an example of a graph over a binary alphabet.

\begin{figure}
\centering
\begin {tikzpicture}[-latex ,auto ,node distance =3 cm and 3cm ,on grid ,
semithick ,
state/.style ={ circle , draw = black , text=black , minimum width =1 cm}]
\node[state] (C)
{$C$};
\node[state] (A) [above left=of C] {$A$};
\node[state] (B) [above right =of C] {$B$};
\path (A) edge [loop left] node[left] {$0$} (A);
\path (C) edge [bend left =] node[below =0.15 cm] {$0$} (A);
%\path (A) edge [bend right = -15] node[below =0.15 cm] {$1/2$} (C);
\path (A) edge [bend left =25] node[above] {$1$} (B);
\path (B) edge [bend left =15] node[below =0.15 cm] {$0$} (A);
\path (C) edge [bend left =15] node[below =0.15 cm] {$1$} (B);
\path (B) edge [bend right = -25] node[below =0.15 cm] {$1$} (C);
\end{tikzpicture}
\caption{Example of a graph with Q = \{A, B, C\} and $\Sigma = \{0, 1\}$.\label{fig:graph}}
\end{figure}

A graph \textit{G} can also be represented as a triple \textit{(Q, E, L)} where \textit{Q} is the set of states, \textit{E} is the set of edges connecting the states and \textit{L} is the set of edges' labels. A walk in a graph \textit{G} is the sequence of labels \textit{l} $\in$ \textit{L} formed by starting at a state \textit{q} $\in$ \textit{Q} and a string \textit{s} that starts as the empty string and going to a next state connected to it by a vertex and appending the vertex's label to \textit{s}. This process can be repeated and when it stops, \textit{s} defines a walk over \textit{G} starting at \textit{q}. Calling the graph of Figure \ref{fig:graph} as \textit{G}, the following is an example of a walk:  start at state \textit{A} and go to \textit{A}, \textit{A}, \textit{B}, \textit{C}, \textit{B}, \textit{A}. This forms the string \textit{s} = 001110.

\begin{definition}[\textbf{Follower Set}]\label{def:followerset}
The follower set of the graph \textit{G} rooted at state \textit{q} $\in$ \textit{Q} is defined as the set of all possible walks that can be formed by starting at state \textit{q}. That is:
\end{definition}

\[ F(q) = \{\omega \in \Sigma^* | q\cdot\omega \in \textit{Q}\}, \]



\noindent where \textit{q$\cdot\omega$} denotes the state arrived after starting at \textit{q} and following the path determined by the word $\omega$.


\begin{definition}[\textbf{Language of a Graph}]\label{def:language}
The language $\mathcal{L}$ of a graph \textit{G} is the the set of follower sets for each state \textit{q} $\in$ \textit{Q}:
\end{definition}

\[ \mathcal{L} = \{F(\textit{q}), \forall \textit{q} \in \textit{Q}\}. \]

A word \textit{w} is called a synchronizing word of \textit{G} if all walks in \textit{G} that generate \textit{w} terminate at the same state and \textit{G} is called \textit{synchronizing} if there exists a synchronization word for every state \textit{q} $\in$ \textit{Q}. 

\section{Graph Minimization}

This section explains the two most widely used algorithms for automata minimization: Moore and Hopcroft.

\begin{definition}[\textbf{Partitions and Equivalence Relations}]\label{def:partition}
Given a set E, a partition of E is a family $\mathcal{P}$ of nonempty, pairwise disjoint subsets of E such that $\bigcup_{P\in\mathcal{P}}P = E $. The index of the partition is its number of elements. The partition defines an equivalence relation on E and the set of all equivalence classes [x], $x\in E$, of an equivalence relation in E defines a partition of the set.
\end{definition}

When a subset \textit{F} of \textit{E} is the union of classes of $\mathcal{P}$ it said that \textit{F} is saturated by $\mathcal{P}$. Given $\mathcal{Q}$, another partition of \textit{E}, it said to be a \textit{refinement} of $\mathcal{P}$ (or that $\mathcal{P}$ is coarser than $\mathcal{Q}$) if every class of $\mathcal{Q}$ is contained by some class of $\mathcal{P}$ and it is written as $\mathcal{Q} \leq \mathcal{P}$. The index of $\mathcal{Q}$ is greater than the index of $\mathcal{P}$.

Given partitions $\mathcal{P}$ and $\mathcal{Q}$ of \textit{E}, $\mathcal{U} = \mathcal{P}\wedge\mathcal{Q}$ denotes the coarsest partition which refines $\mathcal{P}$ and $\mathcal{Q}$. The elements of $\mathcal{U}$ are non-empty sets \textit{P}$\cap$\textit{Q}, \textit{P}$\in\mathcal{P}$ and \textit{Q}$\in\mathcal{Q}$. The notation is extended for multiple sets as $\mathcal{U} = \mathcal{P}_1 \wedge \mathcal{P}_2 \wedge \ldots \wedge \mathcal{P}_n$. When $n=0$,  $\mathcal{P}$ is the universal partition comprised of just \textit{E} and it is the neutral element for the $\wedge$-operation.

Given $F\subseteq E$, a partition $\mathcal{P}$ of \textit{E} induces a partition $\mathcal{P}'$  
of \textit{F} by intersection. $\mathcal{P}'$ is composed by the sets $P\cap F$ with $P\subseteq \mathcal{P}$. If $\mathcal{P}$ and $\mathcal{Q}$ are partitions of \textit{E} and $\mathcal{Q} \leq \mathcal{P}$, the restrictions $\mathcal{P}'$ and $\mathcal{Q}'$ to \textit{F} maintain $\mathcal{Q}' \leq \mathcal{P}'$.

Given partitions $\mathcal{P}$ and $\mathcal{P}'$ of disjoint sets \textit{E} and \textit{E'}, the partition of set $E \cup E'$ whose restriction to \textit{E} and \textit{E'} are $\mathcal{P}$ and $\mathcal{P'}$ is denoted by $\mathcal{P}\vee\mathcal{P}'$. It is possible to write $\mathcal{P} = \vee_{P\vee\mathcal{P}}\{P\}$.

\begin{definition}[\textbf{Irreducible Graph}]\label{def:mingraph}
A graph \textit{G} is said to be irreducible if all its states have distinct follower sets (from Definition \ref{def:followerset}), that is F(p) $\neq$ F(q) for each pair of distinct states p, q $\in$ \textit{Q}.
\end{definition}

From Definition \ref{def:mingraph} it is possible to define an equivalence relation called the Nerode equivalence:

\[
p, q \in Q, p \equiv q \Leftrightarrow F(p) = F(q).
\]

A graph is considered minimal if and only if its Nerode equivalence is the identity. The problem of minimizing a graph is that of computing the Nerode equivalence. The quotient graph \textit{G}/$\equiv$ obtained by taking for \textit{Q} the set of Nerode equivalence classes. The minimal graph is unique and it accepts the same language as the original graph.

Given a set of states \textit{P} $\subset$  \textit{Q} and a symbol $\sigma \in \Sigma$, let $\sigma^{-1}\textit{P}$ denote the set of states \textit{q} such that $\delta(q, \sigma) \in P$. Consider \textit{P}, \textit{R} $\subset$ \textit{Q} and $\sigma \in \Sigma$, the partition of R

\[
(P, \sigma)|R
\]

\noindent the partition composed of two non-empty subsets:

\[
R\cap\sigma^{-1}P = \{r \in R | \delta(r,\sigma) \in P\}
\]

\noindent and

\[
R\backslash\sigma^{-1}P = \{r \in R | \delta(r,\sigma) \notin P\}.
\]

The pair (\textit{P}, $\sigma$) is called a splitter. Observe that (\textit{P}, $\sigma$)|R = {R} if either $\delta(R,\sigma) \subset$ \textit{P}
or $\delta(R,\sigma)\cap P = \emptyset$ and (\textit{P}, $\sigma$)|R is composed of two classes if both $\delta(R,\sigma)\cap P  \neq\emptyset$ and
$\delta(R,\sigma)\cap P^c  \neq \emptyset$  or equivalently if $\delta(R,\sigma) \not\subset P $   and  $\delta(R,\sigma)\not\subset P^c $ . If (\textit{P}, $\sigma$)|R  contains two
classes, then we say that (\textit{P}, $\sigma$) splits R. This notation can also be extended to sequences, using a sequence $\omega \in \Sigma^*$ instead of the symbol $\sigma \in \Sigma$.

\begin{proposition}\label{prop:nerequiv}
The partition corresponding to the Nerode equivalence is the coarsest partition $\mathcal{P}$ such that no splitter (P,$\sigma$), with P $\in \mathcal{P}$ and $\sigma \in \Sigma$, splits a class in $\mathcal{P}$, that is such that (P,$\sigma$)|R = {R} for all P, R $\in \mathcal{P}$ and $\sigma \in \Sigma$.
\end{proposition}

\begin{lemma}\label{lemm:hopcrof}
Let P be a set of states and $\mathcal{P} = {P_1, P_2}$ a partition of P. For any symbol $\sigma$ and for any set of states R, one has:
\end{lemma}
 

\[
(P,\sigma)|R \wedge (P_1, \sigma)|R = (P, \sigma)|R \wedge (P_2, \sigma)|R = (P_1,\sigma)|R \wedge (P_2,\sigma)|R,
\]

\textit{and consequently}

\[
(P,\sigma)|R \geqslant (P_1,\sigma)|R \wedge (P_2,\sigma)|R,
\]
\[
(P_1,\sigma)|R \geqslant (P,\sigma)|R \wedge (P_2,\sigma)|R.
\]
\subsection{Moore's Algorithm}

\begin{algorithm} 
  \caption{Moore(\textit{G})\label{alg:moore}}
    \begin{algorithmic}[1]
      \State $\mathcal{P} \leftarrow InitialPartition(G)$
      \Repeat
      	\State $\mathcal{P}' \leftarrow \mathcal{P}$
      	\ForAll{$\sigma \in \Sigma$}
      		\State $\mathcal{P}_{\sigma} \leftarrow \bigwedge_{P\in \mathcal{P}}(P,\sigma)|Q$
      	\EndFor
      	\State $\mathcal{P} \leftarrow \mathcal{P}\wedge\bigwedge_{\sigma\in\Sigma}\mathcal{P}_{\sigma}$
      \Until{$\mathcal{P} = \mathcal{P}'$}
    \end{algorithmic}
  \end{algorithm}


\subsection{Hopcroft's Algorithm}

\begin{algorithm} 
  \caption{Hopcroft(\textit{G})\label{alg:hop}}
    \begin{algorithmic}[1]
      \State $\mathcal{P} \leftarrow InitialPartition(G)$
      \State $\mathcal{W} \leftarrow \emptyset$
      \ForAll{$\sigma \in \Sigma$}
      	\State Append(($\min$(\textit{F}, $F^c$,$\sigma$),$\mathcal{W}$)
      	\While{$\mathcal{W} \neq \emptyset$}
      		\State (\textit{W},$\sigma$) $\leftarrow$ TakeSome($\mathcal{W}$)
      		\For{each \textit{P}$\in \mathcal{P}$ which is split by $(W,\sigma)$}
				\State \textit{P', P"} $\leftarrow (W,\sigma)|P$      		
				Replace \textit{P} by \textit{P'} and \textit{P"} in $\mathcal{P}$
				\ForAll{$\tau \in \Sigma$}
					\If{$(P,\tau)\in\mathcal{W}$}
						\State Replace $(P,\tau)$ by $(P',\tau)$ and $(P",\tau)$ in $\mathcal{W}$
					\Else
						\State Append(($\min$(\textit{P'}, $P"$,$\tau$),$\mathcal{W}$)				
					\EndIf				
				\EndFor 
      		\EndFor
      	\EndWhile
      \EndFor
    \end{algorithmic}
  \end{algorithm}

\section{Probabilistic Finite State Automata}

\begin{definition}[\textbf{Probabilistic Finite State Automata}]\label{definition:pfsa}
A Probabilistic Finite State Automaton (PFSA) \textit{P} is defined as a quadruple $(Q, \Sigma, \delta, \mathcal{V})$. The first three items are the same as of a graph as defined in Definition \ref{def:graph}, while  $\mathcal{V}$ is a probability function, $\mathcal{V}$: $\delta \rightarrow [0,1)$ which associates a probability to each edge.
\end{definition}

The function $\mathcal{V}$ gives the probabilistic factor to the PFSA. It is the associated probability distribution for the outgoing edges of each state. This means that for each state \textit{q} $\in$ \textit{Q}, there will be a probability $\mathcal{V}(\delta(q, \sigma)), \forall \sigma \in \Sigma$ associated to each edge in such a way that $\Sigma_{\sigma}\mathcal{V}(\delta(q,\sigma)) = 1$ and $0 \leq \mathcal{V}(\delta(q,\sigma)) \leq 1$. The probability associated to an edge is the probability of taking this path once the system is in the state \textit{q}. 

\begin{definition}[\textbf{Morph}]\label{definition:morph}
The probability distribution $\mathcal{V}(q) = \{ \mathcal{V}(\delta(q, \sigma)); \forall \sigma \in \Sigma\}$ of a state is called the state morph.  
\end{definition}

A PFSA can be represented with a graph, in which each $q \in Q$ is represented as a node. The edges are given by function $\delta(q, a) =  q^{\prime}$, indicating there is an edge going from \textit{q} to \textit{$q^{\prime}$} labeled with the symbol \textit{a}. The probability associated with an edge, $\mathcal{V}(\delta(q,a))$ is also in the edge label.

\begin{definition}[\textbf{Deterministic PFSA}]\label{definition:dpfsa}
A PFSA is called deterministic if the outgoing edges of each state are labeled with distinct symbols. 
\end{definition}

An example of a graph of a PFSA is shown in Figure \ref{fig:pfsa}, for which \textit{Q} = $\{A, B, C\}$, $\Sigma = \{0, 1\}$ and the functions $\delta$ and $\mathcal{V}$ are represented in the edges of the graph.

\begin{figure}
\centering
\begin {tikzpicture}[-latex ,auto ,node distance =3 cm and 3cm ,on grid ,
semithick ,
state/.style ={ circle , draw = black , text=black , minimum width =1 cm}]
\node[state] (C)
{$C$};
\node[state] (A) [above left=of C] {$A$};
\node[state] (B) [above right =of C] {$B$};
\path (A) edge [loop left] node[left] {$0/0.25$} (A);
\path (C) edge [bend left =] node[below =0.15 cm] {$0/0.5$} (A);
%\path (A) edge [bend right = -15] node[below =0.15 cm] {$1/2$} (C);
\path (A) edge [bend left =25] node[above] {$1/0.75$} (B);
\path (B) edge [bend left =15] node[below =0.15 cm] {$0/0.2$} (A);
\path (C) edge [bend left =15] node[below =0.15 cm] {$1/0.5$} (B);
\path (B) edge [bend right = -25] node[below =0.15 cm] {$1/0.8$} (C);
\end{tikzpicture}
\caption{A probabilistic version of the graph of Figure \ref{fig:graph}.\label{fig:pfsa}}
\end{figure}

A sequence can be generated by a PFSA as the sequence formed by starting at a given state \textit{q} $\in$ \textit{Q} then following a path with its edges and concatenating the labels for each edge. Its probability is given by multiplying the probability of each edge that was taken.

From Figure \ref{fig:pfsa}, starting at the state \textit{A}, it is possible to form the sequence \textit{u = 1011001} by taking a path going to states \textit{B,A,B,C,A,A} and \textit{B} and concatenating the labels of the path from each of these transitions. By multiplying the probabilities of these edges, it is seen that p(\textit{u}) = $0.75\times0.2\times0.75\times0.8\times0.5\times0.25\times0.75 = 0.0084375$.

It is useful do adapt the concept of synchronization word to the context of PFSA:

\begin{definition}[\textbf{PFSA Synchronization Word}]\label{definition:synchword}
For a state \textit{q} $\in$ \textit{Q}, \textit{w} is a synchronization word if, $\forall \textit{u} \in \Sigma^*$ and $\forall \textit{v} \in \Sigma^*$:

\begin{equation}
\Pr(\textit{u}|\textit{w}) = \Pr(\textit{u}|\textit{vw}).
\label{eq:synchword}
\end{equation}
\end{definition}

Definition \ref{definition:synchword} means that the probability of obtaining any sequence after the synchronization word does not depend on whatever came before \textit{w}. The main problem with this definition is the fact that is not possible to check (\ref{eq:synchword}) for all \textit{u} $\in \Sigma^*$ and for all \textit{v} $\in \Sigma^*$ as there are an infinite number of sequences.

  A solution uses the \textit{d}-th order derived frequency, which is the probability using \textit{u} and \textit{v} from $\Sigma^d$, \textit{d} $\in \mathbb{Z}$, instead of taking them from $\Sigma^*$. Calling $\Pr_d(\omega)$ the d-th order derived frequency of $\omega$, a statistical test (such as the Chi-Squared or Kolmogorov-Smirnov) with significance level $\alpha$ has to be performed with the following null hypothesis for \textit{w} being a synchronization word:

\begin{equation}
\textnormal{Pr}_{d}(\textit{w}) = \textnormal{Pr}_{d}(\textit{uw}), \forall u \in \cup_{i=1}^{L_1}\Sigma^i, \forall d = 1,2,\ldots,L_2,  
\label{eq:practsynchword}
\end{equation}

where $L_1$ and $L_2$ are precision parameters. This means that the statistical test compares the probabilities of words \textit{w} with length from 0 to \textit{$L_2$} with the probabilities of words \textit{uw}, where \textit{u} is a prefix of \textit{w} with lengths from 0 to \textit{$L_1$}. This limits the number of tests to be realized.

A synchronization words is a good starting point to model a system from its output sequence because the probability of its occurrence does not depend on what came before it. Therefore its prefix can be regarded as a transient.

\section{Consolidated Algorithms}

\subsection{D-Markov Machines}

\subsection{CRiSSiS}

%\subsection{Definitions}
%
%\begin{definition}[\textbf{Partitions and Equivalence Relations}]\label{def:partition}
%Given a set E, a partition of E is a family $\mathcal{P}$ of nonempty, pairwise disjoint subsets of E such that $\bigcup_{P\in\mathcal{P}}P = E $. The index of the partition is its number of elements. The partition defines an equivalence relation on E and the set of all equivalence classes [x], $x\in E$, of an equivalence relation in E defines a partition of the set.
%\end{definition}
%
%When a subset \textit{F} of \textit{E} is the union of classes of $\mathcal{P}$ it said that \textit{F} is saturated by $\mathcal{P}$. Given $\mathcal{Q}$, another partition of \textit{E}, it said to be a \textit{refinement} of $\mathcal{P}$ (or that $\mathcal{P}$ is coarser than $\mathcal{Q}$) if every class of $\mathcal{Q}$ is contained by some class of $\mathcal{P}$ and it is written as $\mathcal{Q} \leq \mathcal{P}$. The index of $\mathcal{Q}$ is greater than the index of $\mathcal{P}$.
%
%Given partitions $\mathcal{P}$ and $\mathcal{Q}$ of \textit{E}, $\mathcal{U} = \mathcal{P}\wedge\mathcal{Q}$ denotes the coarsest partition which refines $\mathcal{P}$ and $\mathcal{Q}$. The elements of $\mathcal{U}$ are non-empty sets \textit{P}$\cap$\textit{Q}, \textit{P}$\in\mathcal{P}$ and \textit{Q}$\in\mathcal{Q}$. The notation is extended for multiple sets as $\mathcal{U} = \mathcal{P}_1 \wedge \mathcal{P}_2 \wedge \ldots \wedge \mathcal{P}_n$. When $n=0$,  $\mathcal{P}$ is the universal partition comprised of just \textit{E} and it is the neutral element for the $\wedge$-operation.
%
%Given $F\subseteq E$, a partition $\mathcal{P}$ of \textit{E} induces a partition $\mathcal{P}'$  
%of \textit{F} by intersection. $\mathcal{P}'$ is composed by the sets $P\cap F$ with $P\subseteq \mathcal{P}$. If $\mathcal{P}$ and $\mathcal{Q}$ are partitions of \textit{E} and $\mathcal{Q} \leq \mathcal{P}$, the restrictions $\mathcal{P}'$ and $\mathcal{Q}'$ to \textit{F} maintain $\mathcal{Q}' \leq \mathcal{P}'$.
%
%Given partitions $\mathcal{P}$ and $\mathcal{P}'$ of disjoint sets \textit{E} and \textit{E'}, the partition of set $E \cup E'$ whose restriction to \textit{E} and \textit{E'} are $\mathcal{P}$ and $\mathcal{P'}$ is denoted by $\mathcal{P}\vee\mathcal{P}'$. It is possible to write $\mathcal{P} = \vee_{P\vee\mathcal{P}}\{P\}$.
%
%\begin{definition}[\textbf{Deterministic Automaton}]\label{def:minauto}
%A deterministic automaton over an alphabet $\Sigma$ is a quadruple $\mathcal{A}$ = (Q, i, $\delta$, F). Q is the set of states and $\delta$ is the transition function as in the PFSA from Definition \ref{definition:pfsa}. i $\in$ Q is the initial state from which the walk in this automaton will start. F $\subset$ Q is the set of final states in which the walks can end. For each q $\in$ Q, there is a corresponding subautomaton $\mathcal{A}$ starting in q, called the subautomaton rooted in q or simply automaton at q. 
%\end{definition}
%
%To each state \textit{q} there is a corresponding language $L_q(\mathcal{A}$), which is the set of sequences recognized by the automaton rooted at \textit{q}:
%
%\[
%L_q(\mathcal{A}) = \{w \in \Sigma* | qw \in F\}.
%\]



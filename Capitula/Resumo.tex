Geradores de n�meros pseudo-aleat�rios s�o amplamente utilizados em aplica��es cient�ficas e tecnol�gicas. Particularmente em criptografia, estes s�o empregados em sistemas de chave secreta, 
como geradores de sequ�ncias de cifragem. Neste trabalho, propomos algumas metodologias para o projeto destes geradores a partir de mapas ca�ticos. A primeira � baseada em duas t�cnicas: 
salto de amostras e discretiza��o codificada variante no tempo. Mostra-se que o procedimento possui alta taxa de gera��o de bits por amostra ca�tica quando comparado com a codifica��o 
fixa no tempo, al�m de dispensar p�s-processamento para melhoria de 
suas propriedades  aleat�rias. A outra metodologia utilizada � o emprego de sequ�ncias-m para eliminar a correla��o residual na sequ�ncia codificada. A discretiza��o variante no tempo 
apresenta uma caracter�stica de correla��o bem definida que � aproveitada por um novo bloco de p�s-processamento que utiliza sequ�ncias-m de menor complexidade linear que a metodologia anterior. 
Validam-se os m�todos propostos empregando a bateria de teste NIST.







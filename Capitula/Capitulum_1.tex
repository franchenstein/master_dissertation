\chapter{Introduction}\label{cap:intro}

%the current work, we are interested in modeling discrete dynamical systems having only their outputs. To achieve this goal, we developed an algorithm that analyzes the statistics of output sequences from dynamical systems and model them via Probabilistic Finite State Automata (PFSA). 

%\section{Dynamical Systems and System Modeling}

%Dynamical


{\lettrine[loversize=0.25,findent=0.2em,nindent=0em]{D}{ynamical} systems are mathematical models describing how the state of a system evolves over time. It consists of two main components: a dynamic, which specifies the evolution of the system, and an initial condition from which the system starts \cite{luenberger1979introduction}. They can be either continuous or discrete. Continuous dynamical systems have real numbers as both their inputs and outputs and they can be described by differential equations \cite{strogatz2001nonlinear}. Discrete dynamical systems can be obtained by sampling a continuous one and can only produce symbols from a discrete set called an alphabet and their behavior can be described by difference equations or discrete state transform relations \citep{brianmarcus} \cite{rajagopalan2006symbolic}. 

These systems provide a useful framework for analyzing phenomena in several fields of engineering and science such as electronic circuits \cite{li2008circuit}, machine learning \cite{norgaard2000neural}, meteorology \cite{boulard1999characterization}, mechanics \cite{temam2012infinite} and neurobiology \cite{lewis2005bridging}. These systems often lead to chaotic behavior, which means that given two inputs close to each other produce outputs that greatly diverge from each other, making them difficult to predict and seem almost random even though they are completely deterministic \cite{strogatz2001nonlinear}. 

The aim of systems modeling is to obtain  a simple analytic model  that accurately reflects the statistical description of the system.  This involves two main steps: 

\begin{itemize}
	\item[i.] choosing a class of models capable of representing  the system behavior;
	\item[ii.] developing methods to parameterize the model using  experimental  sequences.
\end{itemize}
\noindent Therefore, we can use the statistics of the model to analyze the behavior  of the system and apply it to system simulation\cite{zeigler2000theory}, performance analysis\cite{li2008circuit} and fault detection \cite{willsky1976survey}. 

In order to obtain models for dynamical systems there are methods and frameworks such as belief networks \cite{heckerman1995learning}, probabilistic context free grammars \cite{corazza2007probabilistic} and hidden Markov models \cite{rabiner1989tutorial} but they tend do be complex and require large sampling times \cite{asok.11}. An alternative method, which is the one chosen in this work, is to use probabilistic finite state automata (PFSA), which can solve these issues and also produce good statistical models.

\section{Probabilistic Finite State Automata}

PFSA can be described as a finite labeled graph with probabilities associated to each edge. As in \cite{asok.11}, we consider the PFSA framework for which symbol generation is probabilistic and the end state is unique, given an initial state and a certain sequence. This differs from the framework presented in \cite{vidal.05} in which the symbol generation probabilities are not specified and there is a distribution over the possible end states. The advantages of using PFSA are that they are simple and the sample time required for learning them is easy to characterize \citep{asok.11} and it is also an efficient framework for learning the causal structure of observed dynamical behavior \cite{murphy1995passively}.

Some PFSA generate sequences with a synchronization word. The statistics of the symbols generated after a synchronization word do not depend on anything that came before it \cite{asok.11}. Thus the synchronization word is deemed to be a good starting point for analysis as anything coming before it can be considered a transient. 

The algorithms that construct PFSA include D-Markov machines \cite{ray2004symbolic}, which are Markov chains of a finite order $D$, meaning it uses the statistics of all subsequences of length $D$ to form its states; the Causal-State Splitting Reconstruction (CSSR) \cite{shalizi2004blind}, which starts by assuming that the systemss being analyzed outputs an independent, identically-distributed sequence with one causal state and splits it to a probabilistic suffix tree of depth $L_{max}$. Each node on the tree defines a state labeled with a suffix and any two nodes are merged if the hypothesis that their next-symbol generation probability is the same according to some statistical test (such as $\chi^2$ or Kolmogorov-Smirnov). 

There is also the Compression via Recursive Identification of Self-Similar Semantics (CRISSiS) \cite{asok.11} which firsts find a synchronization word in the sequence and uses it as a starting point to construct the PFSA. It tests its children (states that contain the synchronization word as prefix) using statistical tests merging states if the test passes and creating new ones when it fails. This is done recursively until an irreducible PFSA is obtained. As it has been shown in \citep{asok.11} CRISSiS outperforms CSSR.



\section{Objectives and Contributions}
 
In the current work, we are interested in modeling discrete dynamical systems having only an observed discrete sequence. To achieve this goal, we developed algorithms that analyze the statistics these sequences and model their systems via PFSA. In order to obtain models that are less memory consuming, our algorithm applies techniques of graph minimization to obtain smaller PFSA.  The first algorithm, ALEPH, is applied to sequences generated by synchronizable systems, i.e. systems that generate synchronization words. The modeling results are compared to other algorithms in the literature that seek similar goals.

As CRISSIS, ALEPH makes use of synchronization words. One contribution of this work is a novel method to find synchronization words which makes use of data structures in order to obtain performance gains over the brute force method used in CRISSIS.

The general structure of the ALEPH algorithm is composed of a few steps, when given an input sequence: 
\begin{itemize}
\item[i.] creating a tree structure with probabilities in which each state represents a subsequence;
\item[ii.] finding the synchronization words;
\item[iii.] group the states in equivalence classes using a statistical criterion;
\item[iv.] applying a graph minimization algorithm to obtain an irreducible PFSA.
\end{itemize}  

\noindent The second algorithm is applied to non-synchronizable systems. It works by following these steps:
\begin{itemize}
\item[i.] construct a D-Markov model for a given $D$;
\item[ii.] group its states in equivalence classes using a statistical criterion;
\item[iii.] measure the average and standard deviation of occurrence of each state for every class;
\item[iv.] divide the $H$ classes with higher standard deviations in three new classes for each of them;
\item[v.] apply a graph minimization algorithm to obtain an irreducible PFSA.
\end{itemize}  

\noindent As non-synchronizable machines lack structures present in their synchronizable counterparts, the second method uses a refinement over the D-Markov model.

\section{Work Structure}

This dissertation is organized in six chapters. Chapter 2 reviews the theoretical background discussing discrete sequences, PFSA and graph minimization algorithm while also showing the CRISSiS and D-Markov Machine algorithms used in the literature. Chapter 3 then presents the ALEPH algorithm and then analyzes the time complexity of running it. Chapter 4 presents some synchronizable dynamical systems and shows the comparative results of ALEPH, CRISSiS and D-Markov when recovering the original PFSA. Chapter 5 shows applications modeled as non-synchronizable dynamical systems and an alternative algorithm to be applied in such situation and how this algorithm perform compared to the ones present in literature. Finally, in Chapter 6 a conclusion is discussed and plans for future works to improve the algorithms are presented.

%%%%%%%%%%%%%%%%%%%%%% FIM DA SE��O - Breve historia do caos %%%%%%%%%%%%%%%%%%%%%%%%%%

%%%%%%%%%%%%%%%%%%%%%% FIM DO CAP�TULO %%%%%%%%%%%%%%%%%%%%%%%%%%%%%%%%%%%%% 


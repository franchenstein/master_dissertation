\chapter{Conclusion}\label{cap:conclus�o}

{\lettrine[loversize=0.25,findent=0.2em,nindent=0em]{I}{n} this work two algorithms for modeling discrete dynamical systems were presented. Both of them use PFSA to obtain compact representations while presenting reasonable time complexities. The first algorithm is ALEPH, which is more suitable to model synchronizable dynamical systems, while the second one is DMGM that covers non-synchronizable PFSA. They both start by analyzing the statistics the output sequence of the system to be modeled and use graph minimization techniques to obtain even more compact results.

The ALEPH algorithm makes use of synchronization words as starting points for its inner workings. A new algorithm was also developed to find synchronization words more efficiently than the brute force method used in CRISSiS. It creates an RTP and explores it until it finds all the synchronization words. After this, it takes the leaf nodes from the RTP and connect them using the $\Omega$ criterion in order to create a complete graph. From this graph a partition is created by grouping states with similar morphs in equivalence classes. Finally, the graph minimization algorithm is then applied to this partition obtaining a final PFSA.

We applied ALEPH to some synchronizable systems such as the ternary even shift, the tri-shift, a six state PFSA and the maximum entropy $(d,k)$-constrained code. For all these cases, ALEPH was able to recover the original machines. Its results for conditional entropy and Kullback-Leibler divergence were compared to the results obtained by CRISSiS and by D-Markov machines and it was seen that a D-Markov machine that compares to an ALEPH generated PFSA would need to be considerably larger, while CRISSiS and ALEPH show similar performance, but ALEPH presents a lower complexity.

The DMGM algorithm starts by creating a D-Markov machine for a given $D$ based on the output sequence from the original system. It then uses the same partitioning used by ALEPH to obtain equivalence classes based on morph similarity. After this, the average and standard deviation of the probability of occurrence of states is taken for each class and the $H$ classes with the highest standard deviations are each split into three: one class containing the states that are $t$ standard deviations above the class average, another for the states that are $t$ standard deviations below average and a final one for the remaining states. This partition is then used as input for a graph minimization algorithm and a final PFSA is obtained.

The DMGM algorithm was applied to two non-synchronizable systems: the logistic map and the fading channel. As this algorithm is actually a refinement method over D-Markov machines and therefore its results were compared to the original D-Markov machines. For the logistic map the DMGM were significantly smaller than the D-Markov machines while retaining similar conditional entropy and Kullback-Leibler divergences for a given $D$. For the fading channel, although the DMGM PFSA are smaller than the D-Markov machines the difference is not as noticeable as for the logistic map, although this difference gets larger as $D$ increases. Still, the DMGM PFSA showed similar performance to their D-Markov counterparts.  This algorithm was also applied to a synchronizable system and it was shown that although it is not capable of recovering the original PFSA like ALEPH it still creates small PFSA that generate sequences similar to the original and still has a significantly reduced amount of states when compared to the respective D-Markov machine for a given $D$. The difference in amount of states that generate good machines for the DMGM varies from $35\%$ in cases where its results are not as good as expected to $99\%$ when applied to the synchronizable system.

\section{Future Work}

Future work that improve the presented algorithms include using techniques from information theory and statistical mechanics to analyze the given sequence from the original system and determine whether it comes from a synchronizable system or not. 

It would also be important to develop more precise criterion to split the high standard deviation classes in the DMGM algorithm, instead of using ad-hoc parameters such as $H$ and $t$.

An improvement over the traditional PFSA would be instead of using fixed probabilities for state transitions in a PFSA, to use a non-deterministic approach to these transitions based on the original sequence statistic.

Finally, it would also be interesting to apply the models obtained by our algorithms to applications such as fault detection in which even smaller and less precise machines are capable of quickly detecting anomalies \cite{asok.14}. 
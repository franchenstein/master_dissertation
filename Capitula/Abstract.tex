Random number generators are widely used in scientific and technological applications. Particularly in cryptography, 
they are used in secret-key systems, such as key sequence generators. In this work, we present two methodologies 
for the design of these generators from chaotic maps. The first one is based on two techniques: Skipping and time-varying 
coded discretization. We show that the proposed method has higher bit generation rate when compared to fixed-time coded 
discretization and dispenses post-processing in order to improve their random properties. Another methodology is the use 
of m-sequences to eliminate the residual correlation of the coded sequence. The time-varying coded discretization has a 
well-defined correlation characteristic that is exploited by a new block of post-processing using m-sequences that 
requires less memory than the previous methodology. The effectiveness of this procedure is verified through the NIST 
test.

